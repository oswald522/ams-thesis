% !TeX root = ../main.tex

\chapter{使用方法}
下面介绍该模版的使用方法.

\section{文件说明}

% 页眉距顶端2.3cm,双实线,页脚距底端2.3cm,单实线,页码在页脚实线下方,格式为“第 1 页” 数字识为阿拉伯,左右各空一格,宋体五号字体。
% 页眉距顶端2.3cm,双实线,页脚距底端2.3cm,单实线,页码在页脚实线下方,格式为“第 1 页” 数字识为阿拉伯,左右各空一格,宋体五号字体。
% 一级标题为第一章(两空格)题目(格式为居中三号黑体、段前1.5行段后1行,多倍行距1.25)
% 二级标题为 “1.1(两空格)题目”(格式为居中宋体四号,段前1.5行段后1行,多倍行距1.25)
% 三级标题为“1.1.1  题目”(格式为黑体小四首行缩进2字符,段前1行段后1行,多倍行距1.25)
% 正文内容(格式为宋体小四,首行缩进2字符,多倍行距1.25)

该模版包含文件结构如表~\ref{tab:file}~所示。

\begin{table}[htb]
  \centering
  \caption{文件说明}
  \begin{tabular}{ll}
    \toprule
    文件名                       & 描述                     \\
    \midrule
    amsthesis.cls   & 模板文件,包括文档和注释   \\
    main.tex                  & 主文件                    \\
    ref.bib                   & BibTeX文件               \\
    data/                     & 包含具体TeX文件的文件夹          \\
    figures/                  & 默认用来存放图片的文件夹           \\
    data/chap00.tex           & 第一章的TeX文件(建议每章一个TeX文件) \\
    data/abstract.tex         & 中英文摘要                  \\
    data/acknowledgements.tex & 致谢                     \\
    data/appendix.tex         & 附录                     \\
    \bottomrule
  \end{tabular}
  \label{tab:file}
\end{table}

\section{插图}

图片通常在 \env{figure} 环境中使用 \cs{includegraphics} 插入,如图~\ref{fig:example}~ 的源代码。建议矢量图片使用 PDF 格式,比如数据可视化的绘图;照片应使用 JPG 格式;其他的栅格图应使用无损的 PNG 格式。

此处可能会涉及到\textcolor{red}{短标题和长标题问题},注意代码的使用 \cs{caption}[短标题,目录当中体现]\{长标题,在正文当中体现\}。此种方式可以将在正文目录标题后面添加补充说明文字,而不在图目录当中体现,表格目录类似。
\begin{figure}[htb]
  \centering
  \includegraphics[width=0.25\textwidth]{figures/demo.png}
  \caption[内容精简出现在表目录]{我的内容非常长。这张图片显示了一个A字母}
  \label{fig:demo}
\end{figure}

\subsection{图表示例}
若图或表中有附注,采用英文小写字母顺序编号,附注写在图或表的下方。
国外的期刊习惯将图表的标题和说明文字写成一段,需要改写为标题只含图表的名称,其他说明文字以注释方式写在图表下方,或者写在正文中。

\textcolor{red}{子图个人通常不使用,都是一张整图当中表明(a),(b),(c),好处是不用调整编辑位置}。
\par 如果一个图由两个或两个以上分图组成时,各分图分别以 (a)、(b)、(c)...... 作为图序,并须有分图题。
推荐使用 \pkg{subcaption} 宏包来处理, 比如图~\ref{fig:subfig-a}~ 和图~\ref{fig:subfig-b}~。
\begin{figure}[htb]
  \centering
  \includegraphics[width=0.5\textwidth]{example-image-a.pdf}
  % \caption*{国外的期刊习惯将图表的标题和说明文字写成一段,需要改写为标题只含图表的名称,其他说明文字以注释方式写在图表下方,或者写在正文中。}
  \caption{示例图片标题}
  \label{fig:example}
\end{figure}

\section{代码表格生成}
\subsection{第一节:代码示例}
以下是一个简单的代码示例:

\begin{lstlisting}[language=Python, caption={Hello World 程序}]
# Hello World 程序
print("Hello, World!")
\end{lstlisting}

\subsection{第二节:另一个代码示例}
以下是另一个代码示例:

\begin{lstlisting}[language=Python, caption={简单的加法程序}]
# 简单的加法程序
a = 5
b = 3
print("a + b =", a + b)
\end{lstlisting}

\section{表格}
生平只用三线表,因此更多表格使用请自定义。

\par 表应具有自明性。为使表格简洁易读,尽可能采用三线表,如表~\ref{tab:three-line}~。
三条线可以使用 \pkg{booktabs} 宏包提供的命令生成。可以调整表格的列间距
\begin{verbatim}\setlength{\tabcolsep}{10mm}
\end{verbatim}或者行距进行优化。
\begin{table}[htb]
  \centering
  \caption{三线表示例}
  \setlength{\tabcolsep}{8mm}     % 调节表格列间距,值不需要特别大5~15mm之间即可
  \begin{tabular}{ll}
    \toprule
    文件名             & 描述               \\
    \midrule
    amsthesis.cls   & 模板文件,包括文档和注释   \\
    ref.bib   & 参考文献             \\
    data文件夹 & 用于编写论文内容 \\
    figures or images & 论文图片目录 \\
    \bottomrule
  \end{tabular}
  \label{tab:three-line}
\end{table}


\section{数学定理}
此部分个人使用的不多,建议自行修改完善,导入熟悉的包。

定理环境的格式可以使用 \pkg{amsthm} 或者 \pkg{ntheorem} 宏包配置。
用户在导言区载入这两者之一后,模板会自动配置 \env{theorem}、\env{proof} 等环境。

\begin{theorem}[Lindeberg--Lévy 中心极限定理]
  设随机变量 $X_1, X_2, \dots, X_n$ 独立同分布, 且具有期望 $\mu$ 和有限的方差 $\sigma^2 \ne 0$,
  记 $\bar{X}_n = \frac{1}{n} \sum_{i+1}^n X_i$,则
  \begin{equation}
    \lim_{n \to \infty} P \left(\frac{\sqrt{n} \left( \bar{X}_n - \mu \right)}{\sigma} \le z \right) = \Phi(z),
  \end{equation}
  其中 $\Phi(z)$ 是标准正态分布的分布函数。
\end{theorem}
\begin{proof}
  Trivial.
\end{proof}






\section{引用}
引用直接使用\cs{cite}命令即可,此处引用文献参考标准\cite{gbt7714-2005}。
疯狂引用一堆参考文献\cite{Gendron1991ComeonplusPA,gbt7714-2005,liende2020SG3}。
\textcolor{red}{模板当中的参考文献条目是悬挂缩进,官方Word当中是首行缩进,美观程度略差一些。}
\section{框架代码}
% 在文档的适当位置
\begin{theorem}[毕达哥拉斯定理]
  在直角三角形中,直角边的平方和等于斜边的平方。
\end{theorem}

\begin{lemma}
  这是一个引理。
\end{lemma}

\begin{corollary}
  这是一个推论。
\end{corollary}

\begin{equation}
  A^2=B^2+C^2
\end{equation}


\section{算法部分}
\begin{algorithm}[htb]
  \caption{How to write algorithms}
  \label{algo:algo2}
  \KwData{this text}
  \KwResult{how to write algorithm with \LaTeX2e }
  initialization\;
  \While{not at end of this document}{
    read current\;
    \eIf{understand}{
      go to next section\;
      current section becomes this one\;
    }{
      go back to the beginning of current section\;
    }
  }
\end{algorithm}

\chapter{论文主要部分的写法}

研究生学位论文撰写,除表达形式上需要符合一定的格式要求外,内容方面上也要遵循一些共性原则。

通常研究生学位论文只能有一个主题(不能是几块工作拼凑在一起),该主题应针对某学科领域中的一个具体问题展开深入、系统的研究,并得出有价值的研究结论。
学位论文的研究主题切忌过大,例如,“中国国有企业改制问题研究”这样的研究主题过大,因为“国企改制”涉及的问题范围太广,很难在一本研究生学位论文中完全研究透彻。



\section{论文的语言及表述}

除国际研究生外,学位论文一律须用汉语书写。
学位论文应当用规范汉字进行撰写,除古汉语研究中涉及的古文字和参考文献中引用的外文文献之外,均采用简体汉字撰写。

国际研究生一般应以中文或英文书写学位论文,格式要求同上。
论文须用中文封面。

研究生学位论文是学术作品,因此其表述要严谨简明,重点突出,专业常识应简写或不写,做到立论正确、数据可靠、说明透彻、推理严谨、文字凝练、层次分明,避免使用文学性质的或带感情色彩的非学术性语言。

论文中如出现一个非通用性的新名词、新术语或新概念,需随即解释清楚。


\section{论文题目的写法}

论文题目应简明扼要地反映论文工作的主要内容,力求精炼、准确,切忌笼统。
论文题目是对研究对象的准确、具体描述,一般要在一定程度上体现研究结论,因此,论文题目不仅应告诉读者这本论文研究了什么问题,更要告诉读者这个研究得出的结论。
例如:“在事实与虚构之间:梅乐、卡彭特、沃尔夫的新闻观”就比“三个美国作家的新闻观研究”更专业、更准确。



\section{摘要的写法}

论文摘要是对论文研究内容的高度概括,应具有独立性和自含性,即应是 一篇简短但意义完整的文章。
通过阅读论文摘要,读者应该能够对论文的研究 方法及结论有一个整体性的了解,因此摘要的写法应力求精确简明。
论文摘要 应包括对问题及研究目的的描述、对使用的方法和研究过程进行的简要介绍、 对研究结论的高度凝练等,重点是结果和结论。

论文摘要切忌写成全文的提纲,尤其要避免“第 1 章……;第 2 章……;……”这样的陈述方式。

\section{引言的写法}

一篇学位论文的引言大致包含如下几个部分:
1、问题的提出;
2、选题背 景及意义;
3、文献综述;
4、研究方法;
5、论文结构安排。
\begin{itemize}
  \item 问题的提出:要清晰地阐述所要研究的问题“是什么”。
    \footnote{选题时切记要有“问题意识”,不要选不是问题的问题来研究。}
  \item 选题背景及意义:论述清楚为什么选择这个题目来研究,即阐述该研究对学科发展的贡献、对国计民生的理论与现实意义等。
  \item 文献综述:对本研究主题范围内的文献进行详尽的综合述评,“述”的同时一定要有“评”,指出现有研究状态,仍存在哪些尚待解决的问题,讲出自己的研究有哪些探索性内容。
  \item 研究方法:讲清论文所使用的学术研究方法。
  \item 论文结构安排:介绍本论文的写作结构安排。
\end{itemize}



\section{正文的写法}

本部分是论文作者的研究内容,不能将他人研究成果不加区分地掺和进来。
已经在引言的文献综述部分讲过的内容,这里不需要再重复。
各章之间要存在有机联系,符合逻辑顺序。



\section{结论的写法}

结论是对论文主要研究结果、论点的提炼与概括,应精炼、准确、完整,使读者看后能全面了解论文的意义、目的和工作内容。
结论是最终的、总体的结论,不是正文各章小结的简单重复。
结论应包括论文的核心观点,主要阐述作者的创造性工作及所取得的研究成果在本领域中的地位、作用和意义,交代研究工作的局限,提出未来工作的意见或建议。
同时,要严格区分自己取得的成果与指导教师及他人的学术成果。

在评价自己的研究工作成果时,要实事求是,除非有足够的证据表明自己的研究是“首次”、“领先”、“填补空白”的,否则应避免使用这些或类似词语。

