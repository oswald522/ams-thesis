% !TeX root = ../main.tex
%

\chapter*{作者在学期间取得的学术成果}
\addcontentsline{toc}{chapter}{作者在学期间取得的学术成果}
同参考文献的写法,在成果最后可附上SCI、EI检索号、影响因子等标志成果学术水平。
% 这一部分的内容直接定义编写吧,毕竟结果可能也不太多
% 已发表(或正式接受)的论文:

% 对于匿名评阅版学位论文,务必删除作者、单位和导师等相关信息。
% 暂不清楚参考文献是否需要进行缩进,需要则调整为label={\hspace{2\ccwd}[\arabic*]
\begin{enumerate}
  \item Dan Wu, Kui Dai, Zhiying Wang. Retargetable Machine-Description System: Multi-layer Architecture Approach. 4th International Conference of on Grid and Cooperative Computing, November/December 2005, Beijing, China. LNCS 3795 Springer-Verlag, 1161~1166, ISSN: 0302-9743. (SCI Index, IDS N.O: BDQ17).
  \item 吴先宇,罗世彬,陈小前等.基于替代模型的高超声速进气道优化[J].弹箭与制导学报,2008,28(1):148-152.
  \item 第一作者.基于替代模型的高超声速进气道优化[J].弹箭与制导学报,2008, 28(1):148-152.
\end{enumerate}


% %对于正式版学位论文,此页放置《作者在学期间取得的学术成果》;
%     \begin{enumerate}
%         \item Dan Wu, Kui Dai, Zhiying Wang. Retargetable Machine-Description System: Multi-layer Architecture Approach. 4th International Conference of on Grid and Cooperative Computing, November/December 2005, Beijing, China. LNCS 3795 Springer-Verlag, 1161~1166, ISSN: 0302-9743. (SCI Index, IDS N.O: BDQ17).
%         \item 吴先宇,罗世彬,陈小前等.基于替代模型的高超声速进气道优化[J].弹箭与制导学报,2008,28(1):148-152.
%         \item 第一作者.基于替代模型的高超声速进气道优化[J].弹箭与制导学报,2008,28(1):148-152.
%     \end{enumerate}
% 以下信息进行注释
\par\textcolor{red}{提醒(正式成文后删除):对于正式版学位论文,此页放置《作者在学期间取得的学术成果》;对于匿名评阅版学位论文,务必删除作者、单位和导师等相关信息。}
